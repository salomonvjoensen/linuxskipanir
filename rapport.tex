\documentclass{article}
\usepackage{graphicx} % Required for inserting images
\usepackage[utf8]{inputenc}
\usepackage[T1]{fontenc}

\title{Rapport í Linux Skipanir}
\author{Salomon Vágadal Joensen, 2022.210\\Jákup Paulason Olsen, 2020.006\\Helena Hentze, 2022.197}
\date{1. Marts 2024}

\begin{document}

\maketitle

\section{Problemformulering}
Hvussu ger man eina kjak heimasíðu sum fólk kunnu vitja og stovna tráðir og leggja innlegg í? Og møguliga eisini hava møguleika at deila media har? 
\begin{itemize}
    \item Man má gera sær greitt at har má vera ein heimasíða, sum fólk vitja.
    \item Á hesari heimasíðuni skal brúkarin kunna síggja kjaksíður
    \begin{itemize}
        \item Fara inn á eina kjak undirsíðu.
        \item Síggja tráðir og kunna stovna tráðir.
        \item Kunna fara inn á einkultar tráðir og svara í einum tráði og viðmerkja navn, tekstsvar og um tey vilja leggja mynd avtrat. 
    \end{itemize}
\end{itemize} 

\section{Tólmenni}
Ein Ubuntu Server við einari lokalari heimasíðu og brúkt ein MariaDB dátagrunn at goyma tráðirnir og postar í. PhpMyAdmin verður brúkt til at síggja dátagrunnin og tað er installera 

\section{Mál / Framgangsháttur}
Vit byrja við einari stutta analysu hvussu hetta skal fremjast.
\begin{itemize}
    \item Arkitektur bygnaða av probleminum og hvussu tað fer at síggja út.
    \item Gera ein databasa í MariaDB har man kann stovna ein \textit{thread} í 3 ymiskum kjakforum har fólk kunna svara uppá.
    \item Við einum fullfíggjaðum MariaDB datagrunn byggja eina heimasíðu sum virkar sum eitt \textit{interface} millum heimasíðuna og datagrunnin.
    \item \textit{Business logic} millumlið verður brúkt PHP til samskifting millum heimasíðuna og MariaDB.
    \item Millumliðið fer at avgera hvussu úrslit frá datagrunninum verður víst.
    
\end{itemize}

\end{document}
